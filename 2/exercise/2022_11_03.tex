\documentclass{article}
\usepackage[utf8]{inputenc}
\usepackage{../../utils/personalmacros}

\title{Algorithmics Exercise 2}
\author{Konstantin Mark}
\date{November 2022}

\begin{document}

\maketitle


\listoftheorems[ignoreall,show={exercise}]

\newpage

\begin{exercise}[Covering the 1's]
    Assume you are given an $n\times m$ matrix $M$ containing only $0$'s and $1$'s. The \textit{lines} of the matrix are its rows and columns; hence the matrix has $n+m$ lines. Show that you can find a smallest set of lines containing all the $1$'s in polynomial time. Illustrate your solution with an example.
\end{exercise}
\begin{solving}
    Denote the columns of the matrix by $C = \{c_1,\dots, c_m\}$, the rows of the matrix by $R = \{r_1,\dots ,r_n\}$. Construct an undirected graph $(G, E)$ with all lines as nodes $G = C\cup R$ and edges $E = \{ (c_i, r_j) : M_{i,j} = 1\}$. Clearly this graph is bipartite. 
    \begin{proposition}
        A set $X\subseteq C\cup R$ is a smallest set of lines containing all the $1$'s if and only if it is a minimal vertex cover of $(G, E)$.
    \end{proposition}
    \begin{proof}
        Let $X$ be a vertex cover, and $i, j$ so that $M_{i,j} = 1$ then by vertex cover property, as the edge $(c_i, r_j)\in E$ either $c_i$ or $r_j$ are in $X$. Similarly, for $X$ a smallest set of lines containing all $1$'s and $e:= (c_i, r_j)\in E$ , then $M_{i,j} = 1$ by definition of $E$. Let $x$ be the line in $X$ that contains this $1$. Then $x\in X$ covers the edge  $e$.\\
        For the equivalence of the minimal sets, note that minimality is defined by cardinality on both the sets of lines containing all $1$'s and the vertex covers.
    \end{proof}
    As finding a minimal vertex cover can be done in polynomial time on bipartite graphs (Corollary of König-Ergerváry, slide 62), we can find a smallest set of lines that covers all $1$'s in polynomial time.

    \begin{example}
        Look at the following matrix: \begin{equation*}
            M = \left(\begin{array}{ccc}
                 1&0&1  \\
                 0&1&0 \\
                 1&0&1
            \end{array}\right)
        \end{equation*}

        This has the following associated bipartite graph:\begin{equation*}
            \begin{tikzcd}
            c_1\arrow[r, no head] \arrow[rdd, no head]&r_1\\
            c_2\arrow[r, no head]&r_2\\
            c_3\arrow[r, no head]\arrow[ruu, no head]&r_3
            \end{tikzcd}
        \end{equation*}
        which has the minimal vertex cover $c_1, r_2, c_3$. 
    \end{example}
\end{solving}
\newpage

\begin{exercise}[Matrix rounding]
Consider the following extension to the feasible matrix rounding problem, where you are given a matrix with $n$ columns, where $n$ is even. In addition to row and column sums, we also consider the sum $S_1$ of the first $n/2$ columns. Argue that there always exists a feasible rounding that, in addition to row sums and column sums, also rounds $S_1$ up or down to an integer $S_1'$, such that $S_1'$ equals the sum of the first $n/2$ columns after the rounding.. Illustrate your solution with an example.    
\end{exercise}
\begin{solving}
Consider the circulation problem $(V,E,\ell, c, d)$ that reduces the normal matrix rounding problem, let $\tilde C :=\{c_1, \dots, c_{n/2}\}$ be the first half of the column nodes. We modify the graph by adding one additional node $s_1$, moving all edges $s\to \tilde c \in \tilde C$ to be $s_1\to \tilde c\in \tilde C$ with the same upper and lower bounds and adding an additional edge $s\to s_1$ with lower bound $\lfloor S_1\rfloor$ and upper bound $\lceil S_1\rceil$. The demand of $s_1$ is - as all other demands in the circulation problem - $0$.

\begin{theorem}
    Feasible extended matrix rounding always exists.
\end{theorem}
\begin{proof}
    As in the normal matrix rounding problem, the original data provides a circulation (including also $S_1$ as the flow from $s$ to $s_1$). Then the integrality theorem says that there is an integral solution and thus a feasible rounding on the extended problem.
\end{proof}

\begin{example}
    Consider the matrix (with the sums already marked in) \begin{equation*}
        M = \left(\begin{array}{cccc|c}
             0.86& 8.22&8.52&0.15& 17.75\\
6.29&0.68&3.85&6.25& 17.07 \\
6.02&6.26&3.24&5.51& 21.03 \\
6.28&4.9&6.54&0.38 &18.1 \\\hline
19.45&20.06&22.15&12.29
        \end{array}\right)
    \end{equation*}
    (with the sums already marked in)
    Then this corresponds to the following graph:
     \begin{equation*}
          \begin{tikzcd}
        &&c_1 \arrow[rr]\arrow[rrd]\arrow[rrdd]\arrow[rrddd]&&r_1\arrow[rrd]&\\
        s\arrow[r, "39|40"] \arrow[rrd, bend right, "22|23"] \arrow[rrdd, bend right, "12|13"']&s_1 \arrow[ru, bend left, "19|20"]\arrow[r, "20|21"]&c_2\arrow[rr]\arrow[rrd]\arrow[rrdd]\arrow[rru]&&r_2\arrow[rr]&& t \arrow[llllll, bend right = 90, "0|\infty"']\\
        &&c_3\arrow[rr]\arrow[rrd]\arrow[rruu]\arrow[rru]&&r_3\arrow[rru]&\\
        &&c_4\arrow[rr]\arrow[rruuu]\arrow[rruu]\arrow[rru]&&r_4\arrow[rruu]&\\
    \end{tikzcd}
     \end{equation*}
\end{example}

\end{solving}
\newpage

\begin{exercise}[Diagonizable matrices]
We consider $n\times n$ matrices where each element is $0$ or $1$. Let $M_{i,j}$ denote the entry of row $i$ and column $j$. We call a matrix \textit{diagonalized} if $M_{i,i}=1$ for $i = 1, \dots, n$. We call a matrix \textit{diagonalizable} if we can obtain a diagonalized matrix by permuting its columns and/or permuting its rows. \\
For instance, the following matrix is diagonalizable, as we can swap the second and the third row, and then swap the first and the second column.\begin{equation*}
    M = \left(\begin{array}{ccc}
         0&1&1  \\
         0&0&1\\
         1&0&0
    \end{array}\right)
\end{equation*}
Explain how to test in polynomial time (in $m$) whether a matrix is diagonalizable. \\
Solve the problem by formulating it in terms of a perfect matching problem in a bipartite graph with $n+n$ vertices.
\end{exercise}
\begin{solving}
    Denote the columns of the matrix by $C = \{c_1,\dots, c_n\}$, the rows of the matrix by $R = \{r_1,\dots ,r_n\}$. Construct an undirected graph $(G, E)$ with all lines as nodes $G = C\cup R$ and edges $E = \{ (c_i, r_j) : M_{i,j} = 1\}$. Clearly this graph is bipartite. 
    \begin{proposition}
        A matrix $M$ is diagonisable if and only if the associated graph has a perfect matching.
    \end{proposition}
    \begin{proof}
        Permutating columns and rows does not change anything about the topology of the associated graph, except for renaming the nodes (but staying in the respective partition). Thus, the diagonal entries in the diagonalised matrix correspond (wrt. the permutations) to the perfect matchings in the in the associated graph.
    \end{proof}
    As we can find bipartite matching in polynomial time (maximally $O(n^3)$). 


    \begin{example}
        The example matrix given above has the associated graph
        \begin{equation*}
            \begin{tikzcd}
            c_1\arrow[rdd, no head]&r_1\\
            c_2\arrow[ru, no head]&r_2\\
            c_3\arrow[ruu,  no head]\arrow[ru, no head]&r_3
            \end{tikzcd}
        \end{equation*}
        The permutations described above lead to the following graph: 
        \begin{equation*}
            \begin{tikzcd}
            c_1\arrow[rdd, no head]&r_1\\
            c_2\arrow[ru, no head]&r_2\\
            c_3\arrow[ruu,  no head]\arrow[ru, no head]&r_3
            \end{tikzcd} 
            \to 
            \begin{tikzcd}
            c_1\arrow[rd, no head]&r_1\\
            c_2\arrow[ru, no head]&r_3\\
            c_3\arrow[ruu,  no head]\arrow[r, no head]&r_2
            \end{tikzcd}
            \to 
            \begin{tikzcd}
            c_2\arrow[r, no head]&r_1\\
            c_1\arrow[r, no head]&r_3\\
            c_3\arrow[ruu,  no head]\arrow[r, no head]&r_2
            \end{tikzcd}
        \end{equation*}
        from which the perfect matchings can be easily read.
    \end{example}
\end{solving}
\newpage

\begin{exercise}[Significant Edges]
    An edge $(u,v)\in E$ of a flow network $N= (V,E,c,s,t)$ is \textit{significant} if there is a minimum cut $(A,B)$ such that $u\in A$ and $v\in B$. Design a polynomial-time algorithm that finds all significant edges of a given flow network. You may assume that all capacities are positive integers. Argue for correctness and polynomial running time of the algorithm. 
\end{exercise}
\begin{solving}
Let $f$ be a maximal flow in $N$.
    Let $\tilde e\in E$ be arbitrary. Define $\tilde N$ as the flow network with $\tilde c(\tilde e) = c(\tilde e)-1$. Calculate a maximal flow $\tilde f$ of $\tilde N$.
    \begin{proposition}
        $v(\tilde f) = v(f)-1$ if and only if $\tilde e$ crosses a minimal cut $(A,B)$
    \end{proposition}
    \begin{proof}
        Let $A,B$ be the minimal cut. By the max-flow min-cut theorem, $\fcap_N(A,B) = v(f)$ and therefore $\fcap_{\tilde N}(A,B) = v(f)-1$. Then $$f(e) = \begin{cases}
            f(e)-1& e = \tilde e\\
            f(e) &\text{otherwise}
        \end{cases}$$ is a flow with $v(\tilde f) = v(f)-1 = \fcap_{\tilde N}(A,B)$. Thus $\tilde f$ is a maximal flow in $\tilde N$ and all maximal flows in $\tilde N$ will have the value $v(f)-1$.
        \\
        Let on the other hand $v(\tilde f) = v(f)-1$. This means we have a minimal cut $(A,B)$ with $\fcap_{\tilde N}(A,B)\geq \fcap_N(A,B)-1$. If $e$ does not cross $(A,B)$ then $\fcap_{\tilde N}(A,B)= \fcap_N(A,B) = v(f)-1$, which is a contradiction to minimality and the assumption, so $e$ must cross $(A,B)$ and $\fcap_N(A,B) = v(f)$.
    \end{proof}
    Thus we can find out all significant edges in $|E|+1$ max flow calculations. Therefore the running time is $O((m+1)m^2\log(C) = O(m^3\log(C))$.
\end{solving}
\newpage

\begin{exercise}[Acyclic Flows]
    Consider a flow network $N= (V,E,c,s,t)$. A flow $f$ in $N$ is called \textit{acyclic} if the digraph $G= (V,E')$ with $E'= \{e\in E: f(e)>0\}$ is acyclic. \begin{enumerate}
        \item Give an example of a flow which is not acyclic, and 
        \item show that every flow nework has a maximum flow that is acyclic.
    \end{enumerate}
    Hint: start with any maximum flow and make it acyclic.
\end{exercise}
\begin{solving}
    \begin{enumerate}
        \item Consider the following flow network:\begin{equation*}
            \begin{tikzcd}
                s \arrow[r, "10"]& x\arrow[d, bend right, "3"']\arrow[r, "10"] &t\\
                &y\arrow[u, bend right, "3"']
            \end{tikzcd}
        \end{equation*}
        Then clearly the the flow \begin{equation*}
            \begin{tikzcd}
                s \arrow[r, "10"]& x\arrow[d, bend right, "3"']\arrow[r, "10"] &t\\
                &y\arrow[u, bend right, "3"']
            \end{tikzcd}
        \end{equation*}
        (i.e. just taking maximal capacity everywhere) is not acyclic, whereas the flow 
        \begin{equation*}
            \begin{tikzcd}
                s \arrow[r, "10"]& x\arrow[d, bend right, "0"']\arrow[r, "10"] &t\\
                &y\arrow[u, bend right, "0"']
            \end{tikzcd}
        \end{equation*}
        is.
        \item Let $f$ be a flow that is not acylic. Let $(n_1,n_2,\dots,n_{k-1}, n_1)$ be a cycle in $(V,E')$. Let $q:= \min\{f((n_j,n_{j+1})): 1\leq j\leq k-1\}$. By definition, $q>0$. Define $\tilde f$ to be like $f$ except for the edges on the cycle, where it is reduced by $q$. As $t$ can by definition not be on the cycle (no edges going from there), the value of this adjusted flow is the same. However, the graph $(V,E'')$ with $E'':=\{e\in E: \tilde f(e)>0\}$ has at least one cycle less than $E'$. As the number of cycles in a finite graph must be finite, this procedure ends after a finite number of steps with a maximal flow that is acyclic. 
    \end{enumerate}
\end{solving}
\newpage

% Unfinished
\begin{exercise}[Unique Flows]
    Let $N=(V,E,c,s,t)$ be a flow network such that $(V,E)$ is acyclic and all capacities are positive integers. Let $m= |E|$. Describe a polynomial-time algorithm that checks whether $N$ has a unique maximum flow, by solving $\leq m+1$ max-flow problems. Argue for correctness and polynomial running time of the algorithm.
\end{exercise}
\begin{solving}
    We first calculate a maximal flow $f$. Consider now for any $\tilde e\in E$ the flow network $N_{\tilde e} = (V,E,c_{\tilde e}, s, t)$ with \begin{equation*}
        c_{\tilde e}: \begin{cases}
            E & \to \mathbb N\\
            c_{\tilde e}(e) &\mapsto\begin{cases}
                c(e)-1 & e = \tilde e\\
                c(e) & \text{otherwise}
            \end{cases}
        \end{cases}.
    \end{equation*}
    Then if there is a  maximal flow $f_{\tilde e}$ for $N_{\tilde e}$ with $v(f_{\tilde e}) = v(f)$, then the maximal flow is not unique. If we cannot find such a maximal flow, $f$ is unique. \\
    Since we do $m+1$ max flow problem solves, the running time is $O(m^3\log(C))$.
\end{solving}
\newpage



\begin{exercise}[$k$-Edge Partitions]
    Let $k$ be a positive integer. A \textit{$k$-edge partition} of a graph $G =(V,E)$ is a partition of $E$ into $k$ (possibly empty) sets $E_1,\dots, E_k$ such that every vertex $v\in V$ is incident to at most one edge from each set $E_i$, $1\leq i\leq k$. Show that every $k$-regular bipartite graph has a $k$-edge partition, and such a partition can be found in polynomial time (Hint: use induction on $k$).
\end{exercise}
\begin{solving}
    We conduct the proof by induction:\begin{enumerate}
        \item[$k=1:$] Every vertex is incident to only one edge, $E = E_1$.
        \item[$k = n+1$] Assume that for any graph $k=n$ we can find such a partition. Consider a perfect matching $M$ of $G$ (this exists due to $k$-regularity and Könige-Frobenius). Let $\tilde G = (V,E\backslash M)$. Then every node in $\tilde G$ has exactly $\underbrace{n+1}_{\text{in } G}-\underbrace{1}_{M \text{ is perfect matching}} = n$ neighbours. By setting $E_{n+1}:= M$, this results in a $n+1$ edge partition for $G$, since every $v$ is incidend to exactly one $m\in M$.
    \end{enumerate}
The running time of this partitioning algorithm is \begin{equation*}
    \sum_{i=1}^k O(|E_i|\cdot n) = \sum_{i=1}^k O (\frac{i\cdot n}2 n) = O ((k^2+k)\cdot n^2) = O(m^2)
\end{equation*} since $m = \frac{kn}2$ and therefore $n = O(m)$.
\end{solving}

\newpage
\begin{exercise}[Cute Subsets]
    Let $A_1,\dots, A_m$ be nonempty subsets of a set $S$ with $n$ elements. Call a sequence of $m$ \textit{distinct} elements $a_1,\dots, a_m\in S$ cute if $a_i\in A_i$ for $1\leq i\leq m$. Show that a cute sequence can be found (if it exists) in time that is polynomial in $n+m$.
\end{exercise}
\begin{solving}
Let without loss of generality $S = \{1, \dots, n\}$ and consider the graph $G:= (A\cup S, E)$ with $A := \{A_1, \dots, A_m\}$ and edges $E:= \{(A_i, j): j\in A_i\}$. \begin{equation*}
    \begin{tikzcd}
        A_1\arrow[r, no head]\arrow[rddd, no head]& 1\\
        A_2\arrow[r, no head]\arrow[rd, no head]& 2 \\
        \vdots\arrow[rd, no head] & \vdots\\
        A_m \arrow[ru, no head] & n
    \end{tikzcd}
\end{equation*}
Clearly, $G$ is bipartite. With Ford-Fulkerson, we can find a maximal matching in time $O(|E|\cdot|V|)\leq O(mn(m+n)) \leq O((m+n)^3)$. A maximal matching $M$ of this graph satisfies $|M|\leq \min\{m,n\}$. If $|M| = |A|$, then for every $i$ there is exactly one edge of the form $(A_i, j_i)$ in $M$ and the sequence $j_i$ is cute.
    
\end{solving}

\end{document}